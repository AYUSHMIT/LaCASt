\documentclass{article}
\usepackage[utf8]{inputenc}
\usepackage{hyperref}
\usepackage{amsmath}
\usepackage{listings}
\usepackage[dvipsnames]{xcolor}
\lstset{
    language=[LaTeX]Tex,
    showspaces=true,
    keywordstyle=[1]\color{Red},
    keywordstyle=[2]\color{Orange},
    keywordstyle=[3]\color{Sepia},
    keywordstyle=[4]\color{PineGreen},
    keywordstyle=[5]\color{Blue},
    keywords=[1]{P,JacobiP,cos,cos@,Cos},
    keywords=[2]{alpha,Alpha},
    keywords=[3]{beta,Beta},
    keywords=[4]{n},
    keywords=[5]{a,Theta},
    xleftmargin=.2\textwidth, 
    xrightmargin=.2\textwidth
}

\begin{document}
\begin{center}
    \Large
    \LaTeX-Equation 2 CAS\\[12pt]
    \large Examples
\end{center}

\section{The Jacobi Polynomial}
Let us take a look to the Jacobi polynomial (\href{http://dlmf.nist.gov/18.3\#T1.t1.r2}{\textcolor{blue}{\underline{DLMF-Link}}}). We want to translate the general expression (section \ref{sec:generic}) to a CAS (section \ref{sec:cas}) representation. For instance we have the following term:
\begin{equation}\label{eq:jacP}
    P_n^{(\alpha, \beta)}(\cos(a \Theta))
\end{equation}

\subsection{Generic \LaTeX}\label{sec:generic}
\begin{lstlisting}[mathescape]
P_n^{(\alpha,\beta)}(\cos(a\Theta))
\end{lstlisting}

\subsection{Semantic \LaTeX}\label{sec:semantic}
\begin{lstlisting}[mathescape]
\JacobiP{\alpha}{\beta}{n}@{\cos@{a\Theta}}
\end{lstlisting}

\subsection{Computer Algebra Systems (CAS)}\label{sec:cas}
From now on, be aware of the spaces between the symbols. For instance, a space between $a$ and $\Theta$ is needed to show $a$ and $\Theta$ are different variables. Otherwise Maple (for example) would think $aTheta$ is one new variable.
\subsubsection{Maple}\label{subsec:maple}
\begin{lstlisting}[mathescape]
JacobiP(n,alpha,beta,cos(a Theta))
\end{lstlisting}

\subsubsection{Mathematica}\label{subsec:mathematica}
\begin{lstlisting}[mathescape]
JacobiP[n,\[Alpha],\[Beta],Cos[a \[Theta]]]
\end{lstlisting}
\end{document}
